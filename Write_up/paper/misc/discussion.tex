% !TEX root = ../main.tex
\section{Discussion}
\label{sec:conc}

In this paper we have presented the HMC algorithm, we have introduced the reader to a FOPPL and have explained how we merge together FOPPL programs and HMC inference. We are currently expanding the HMC algorithm for use with both discrete and continuous parameters. This does require the design of new integrator, which has been proposed by \citep{nishimura2017discontinuous}, however, it has never been implemented for a probabilistic programming language, to the authors knowledge.  
%Fourth level headings must be flush left, initial caps, bold, and
%Roman type.  Use one line space before the fourth level heading, and
%place the section text immediately after the heading with no line
%break, but an 11 point horizontal space.

%\subsection{CITATIONS, FIGURES, REFERENCES}
%
%
%\subsubsection{Citations in Text}
%
%%Citations within the text should include the author's last name and
%%year, e.g., (Cheesman, 1985). References should follow any style that
%%you are used to using, as long as their style is consistent throughout
%%the paper.  Be sure that the sentence reads correctly if the citation
%%is deleted: e.g., instead of ``As described by (Cheesman, 1985), we
%%first frobulate the widgets,'' write ``As described by Cheesman
%%(1985), we first frobulate the widgets.''  %Be sure to avoid
%%%accidentally disclosing author identities through citations.
%
%\subsubsection{Footnotes}
%
%%Indicate footnotes with a number\footnote{Sample of the first
%%  footnote.} in the text. Use 8 point type for footnotes. Place the
%%footnotes at the bottom of the column in which their markers appear,
%%continuing to the next column if required. Precede the footnote
%%section of a column with a 0.5 point horizontal rule 1~inch (6~picas)
%%long.\footnote{Sample of the second footnote.}
%
%\subsubsection{Figures}


%Make sure that the figure caption does not get separated from the
%figure. Leave extra white space at the bottom of the page rather than
%splitting the figure and figure caption.
%\begin{figure}[ht]
%	\vspace{.3in}
%	\centering
%	\includegraphics[width=.3\textwidth]{weierstrass_si_small/resized002.png}\quad
%	\includegraphics[width=.3\textwidth]{weierstrass_si_small/resized003.png}
%	\vspace{.3in}
%	\caption{A prediction of the Weierstrass on 20\% of the data set, 200 pts, in the single-input model. The top row is a linear combination of covariance functions operating on the original observations. The middle row represents the prediction, the posterior mean is in green, the observables are in red and the targets are in blue. The bottom row is the posterior covariance at the levels 0 and 4.}
%	\label{pics:si_wie_s}
%\end{figure}
%\begin{figure}[h]
%\vspace{.3in}
%\centerline{\fbox{This figure intentionally left non-blank}}
%\vspace{.3in}
%\caption{Sample Figure Caption}
%\end{figure}
%
%\subsubsection{Tables}
%
%All tables must be centered, neat, clean, and legible. Do not use hand-drawn tables. Table number and title always appear above the table.
%See Table~\ref{sample-table}.
%
%Use one line space before the table title, one line space after the table title, and one line space after the table. The table title must be
%initial caps and each table numbered consecutively.
%
%%\begin{table}
%%\caption{Sample Table Title} \label{sample-table}
%%\begin{center}
%%\begin{tabular}{ll}
%%{\bf PART}  &{\bf DESCRIPTION} \\
%%\hline \\
%%Dendrite         &Input terminal \\
%%Axon             &Output terminal \\
%%Soma             &Cell body (contains cell nucleus) \\
%%\end{tabular}
%%\end{center}git 
%%\end{table}